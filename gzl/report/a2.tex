\documentclass{article}

\usepackage{fullpage}
\usepackage[latin1]{inputenc}
\usepackage[danish]{babel}
\usepackage{listings}
\usepackage{caption}
\usepackage[table]{xcolor}
\usepackage{amssymb}
\usepackage{amsmath}
\usepackage{fancyhdr}
\usepackage{lastpage}
\usepackage{parskip}
\usepackage{abstract}
\usepackage{url}
\usepackage{float}
\usepackage{enumitem}
\usepackage[all]{xy}
\usepackage{amstext}
\usepackage{fancybox}
\usepackage{amsmath}
\usepackage{graphicx}
\usepackage{subfigure}
\usepackage[bottom]{footmisc}
\usepackage{hyperref}
\usepackage{tikz}

% bootstrap label style highlighting
\newcommand\hw[2][]{\tikz[overlay]\node[fill=blue!20,inner sep=1pt, anchor=text, rectangle, rounded corners=0.1mm,#1] {#2};\phantom{#2}}

% styling
\newcommand{\code}[1]{\texttt{#1}}

% diagrams
\newcommand{\switch}[1]%
  {\ovalbox{\text{\begin{minipage}{1.2in}\centering #1\end{minipage}}}}
\newcommand{\minibox}[1]%
  {\ovalbox{\text{\begin{minipage}{0.85in}\centering #1\end{minipage}}}}

\pagestyle{fancy}
\fancyhf{}
\setlength{\parindent}{0pt}
\setlength{\headheight}{15pt}
\setlength{\headsep}{25pt}
\lfoot{Side \thepage{} af \pageref{LastPage}}
\rfoot{30/09-2013}
\lhead{Embedded Systems}
\chead{Assignment 1}
\rhead{}

\title{Assignment 2}
\date{11.11.2013}
\author{
  Simon Altschuler\\
  \code{s123563}
  \and
  Markus F�revaag\\
  \code{s123692}
}

\begin{document}
\maketitle
\centerline{Gruppens arbejde har v�ret fordelt lige i forbindelse med udarbejdelse
af opgaven og rapporten.}
\clearpage

\tableofcontents
\clearpage

\section{Introduktion}
Vi har i denne opgave analyseret og oversat et specifikt filter fra den tidligere \code{C} implementation. Dette er MWI filteret
Lavet C til assembler, udviklet hardware

\section{Problemstilling}
Performance overvejelser, generelt vs. problem-specifikt (finde en balance), problemer ved at g� fra C til assembler
\subsection{Moduler}
Controllers ansvar

\section{Design}
\subsection{Instruktionss�t}
Vores instruktionss�t best�r af 13 forskellige instruktioner, hvoraf 9 er brugt i implementationen af MWI filteret. Grunden til de ekstra instruktioner er at vi gerne ville implementere et fuldt funktionelt instruktionss�t, samt at vi med de ekstra instruktioner kunne pr�ve forskellige metoder til implementation af MWI filteret.

\begin{table}[H]
  \begin{tabular}{|l|l|l|}
    \hline
    \rowcolor{blue!25} Instruktion & Argumenter & Beskrivelse \\ \hline
    \code{set}   & \code{ra}, \code{imm} 
    & S�t register \code{ra} til den konstante v�rdi \code{imm} \\ \hline
    \code{cmp}   & \code{ra}, \code{rb} 
    & Sammenlign register ra og rb vha.\ en \code{sub} operation og s�t flag i ALU'en  \\ \hline
    \code{mov}   & \code{ra}, \code{rb} 
    & Kopier v�rdi i register \code{rb} til \code{ra} \\ \hline
    \code{addi}  & \code{ra}, \code{rb}, \code{simm} 
    & Adder register \code{rb} og den konstante v�rdi \code{simm} og gem resultatet i register \code{ra} \\ \hline
    \code{add}   & \code{ra}, \code{rb}, \code{rc} 
    & Adder register \code{rb} og \code{rc} og gem resultatet i register \code{ra} \\ \hline
    \code{sub}   & \code{ra}, \code{rb}, \code{rc} 
    & Subtraher register \code{rb} og \code{rc} og gem resultatet i register \code{ra} \\ \hline
    \code{mul}   & \code{ra}, \code{rb}, \code{rc} 
    & Multiplicer register \code{rb} og \code{rc} og gem resultatet i register \code{ra} \\ \hline
    \code{div}   & \code{ra}, \code{rb}, \code{rc} 
    & Divider register \code{rb} og \code{rc} og gem resultatet i register \code{ra}. Se \ref{division_method} \\ \hline
    \code{load}  & \code{ra}, \code{rb} 
    & Load data til \code{ra} fra RAM med adressen i register \code{rb} \\ \hline
    \code{store} & \code{ra}, \code{rb} 
    & Lagr data i \code{rb} i RAM p� adressen i register \code{ra} \\ \hline
    \code{jmp}   & \code{block} 
    & Branch til blok med id \code{block} \\ \hline
    \code{jlt}   & \code{block} 
    & Branch til blok med id \code{block} hvis ALU flag neg er h�jt \\ \hline
    \code{jgt}   & \code{block} 
    & Branch til blok med id \code{block} hvis b�de ALU flag neg og zero er lavt \\ \hline
  \end{tabular}
  \caption{Instruktionss�ttet. \code{simm} st�r for small immediate, da denne er 3 bits kortere end \code{imm}}
\end{table}

\subsection{Moduler}
Vi har lavet hver komponent som en Gezel \code{dp} og de er alle forbundet i \code{Platform.fdl}'s CPU \code{dp}. Den eneste undtagelse er \code{inst\_parser.fdl} som bruges i controlleren via en \code{use} statement, for at abstrahere de forskellige fysiske dele af instruktionerne v�k.

\subsection{CPU diagram}

\section{Implementering}
Brugt ternaries i stedet for mux komponent fleste steder

\subsection{Assembler}
Vi har valgt at udvikle en cross-compiler i \code{Ruby} for at lette arbejdet med udviklingen af assembler kode, og for at komme t�ttere et realistisk fuldendt software/hardware samspil. Programmet parser assembler meget lig \code{x86} og outputter \code{Gezel}-l�sbar data i hex format. Det har v�ret en stor hj�lp at have dette program ved h�nden da skriv-test-gentag workflowet har v�ret meget bekvemt. Da dette har v�ret en del af opgaven per se, vil vi ikke g� i detaljer med implementeringen. Den er dog inkluderet i kildekoden og ligger i filen \code{assembler.rb}.

\subsection{Pre-processor}
Vi har gjort brug af \code{C} pre-processoren til inkludering af filer i \code{Gezel} koden. Vi har udviklet hvert komponent i sin egen fil og samlet dem med \code{\#include} statements. Det har gjort det meget mere overskueligt at arbejde med end at have alt st�ende i �n lang fil.

\subsection{Komponenter}
\begin{table}[H]
  \begin{tabular}{|l|p{14cm}|}
    \hline
    \rowcolor{blue!25} Komponent & Beskrivelse \\ \hline
    Controller  & Styrer kontrol signaler til de fleste andre komponenter, s�som write-flag, jump-flag og ALU selector.  \\ \hline
    Program Counter  & Styrer hvilken adresse der l�ses fra instruction memory  \\ \hline
    Jump Handler  & Holder styr p� om der skal udf�res et jump vha. ALU flag og kontrolsignaler  \\ \hline
    Inst. memory & ROM, som indeholder program data \\ \hline
    ALU & Klassisk ALU komponent, udf�rer addition, subtraktion og multiplikation. S�tter \code{neg} og \code{zero} flag afh�ngig af resultatet. \\ \hline
    Register File & Indeholder 8 stk. 32-bit registre og kan outputte to af deres v�rdier per cycle. Kan skrive �n 32-bit v�rdi til et vilk�rligt register per cycle. \\ \hline
  \end{tabular}
  \caption{Komponenter i CPU'en}
\end{table}

\subsubsection{Division}
Gezel kan ikke udf�re division hvorfor vi har m�ttet implementere dette p� anden vis. Vi har haft valget mellem at lave division implementeret i assembler med en l�kke og en counter, eller at bruge bit shifting og en prekalkuleret konstant. Vi har valgt den sidstn�vnte da dette g�r assembler koden langt simplere og det bruger kun en enkelt cycle i forhold til assembler metoden hvis k�retid stiger lin�rt med resultatets st�rrelse.

\section{Resultater}

\section{Profiling}

\subsection{Speed}
\subsection{Area}
\subsection{Power}

\section{Konklusjon}

\section{K�rsel}
Hvordan bruke run script osv.

\newpage
\appendix
\section{Clang}\label{app:clang_asm}
Syg kode her

\end{document}
