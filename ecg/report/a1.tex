\documentclass{article}

\usepackage{fullpage}
\usepackage[latin1]{inputenc}
\usepackage[danish]{babel}
\usepackage{listings}
\usepackage{caption}
\usepackage{subcaption}
\usepackage{xcolor}
\usepackage{amssymb}
\usepackage{amsmath}
\usepackage{fancyhdr}
\usepackage{lastpage}
\usepackage{hyperref}
\usepackage{parskip}
\usepackage{graphicx}
\usepackage{abstract}
\usepackage{url}
\usepackage{float}

\newcommand{\code}[1]{\texttt{#1}}

\pagestyle{fancy}
\fancyhf{}
\setlength{\parindent}{0pt}
\setlength{\headheight}{15pt}
\setlength{\headsep}{25pt}
\lfoot{Side \thepage{} af \pageref{LastPage}}
\rfoot{30/09-2013}
\lhead{Embedded Systems}
\chead{Assignment 1}
\rhead{}

\title{Assignment 1}
\date{30.09.2013}
\author{
  Simon Altschuler\\
  \code{s1236563}
  \and
  Markus F�revaag\\
  \code{s123692}
}

\begin{document}
\maketitle
\clearpage

\tableofcontents
\clearpage

\section{Introduktion}
Denne opgaver omhandler processering af signaler fra et Elektrokardiogram apparat (herefter ECG). Form�let er omdanne de r� signaler til filtreret data, som kan bruges til at m�le puls og sp�nding, og advare om forest�ende problemer hos patienten.

Data fra hardwaren er simuleret ved at l�se linier af tal fra en tekstfil, s�ledes at rigtig data i princippet kunne bruges uden at �ndre andet end funktionen der henter et nyt sample.

\subsection{Problemstilling}
Udfordringen i denne opgave er at implementere signalfiltre og detektere egenskaber effektivt og struktureret, samt at pr�sentere dataen for brugeren p� en hensigtsm�ssig og brugbar facon.

Datas�ttene har op til flere millioner samples, og det er derfor vigtigt at implementere datastrukturer og algoritmer p� en m�de som kan h�ndtere arbitr�t store datas�t, mens ydeevnen forbliver god.

\section{Analyse}
\subsection{Funktioner}
\subsubsection{Sensor}
\subsubsection{Filtre}
\subsubsection{Peak detektion}
\subsubsection{Output}


\section{Design}
\subsection{Arkitektur}
I programmet har vi lagt til filene \code{sensor.c}, \code{filter.c}, \code{peak\_detect.c}, \code{output.c} og \code{display.c}, samt deres tilhoerende header-filer. I \code{sensor.c} ligger alt som har med aa innlese raw data fra en simulert ECG maskin. \code{filter.c} filtrer saa denne dataen vha. de filtre der er beskrevet under Funktioner. Der efter blir dataen gitt over til \code{peak\_detect.c} som ser efter peaks, puls og eventuelle abnormaliteter i hjerterytmen. Denne informasjonen, i tillegg til den filtrerede data, blir saa outputtet eller displayed med enten \code{output.c} eller \code{display.c}.

Vi mener dette gir en oversiktelig inndeling av programmet da hver fil har, i henhold med navnet, er oppdelt ut i fra selve harware strukturen til en ECG maskin. Man kan derfor nemt foelge prosessen, helt fra signalet blir avlest, filtrert, prosessert og til slutt vist til brukeren.

\subsection{Sensor}
\subsection{Filtre}
\subsection{Peak detektion}
\subsection{Output}

\section{Implementering}

\section{Resultater}

\section{Diskussion}

\section{Konklusion}

\end{document}
